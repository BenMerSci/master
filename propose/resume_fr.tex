\documentclass[12pt]{article}
\usepackage[T1]{fontenc}
\usepackage[utf8]{inputenc}
\usepackage[numbers]{natbib}
\usepackage{url}
\usepackage{graphicx}
\usepackage[doublespacing]{setspace}
\usepackage{geometry}


\begin{document}

\begin{titlepage}
    \begin{center}

        \vspace*{-6em}
        
        \Large{\textbf{Prédiction des forces d'interaction au sein de réseaux d'interactions trophiques}}\\

        \vspace{3em}

        {Résumé du projet}\\
        \vspace{3em}
        \includegraphics[scale=0.65]{udes_logo.jpg}\\
        \vspace{3em}

        \normalsize{\textbf{Présenté à:}\\
        Dominique Gravel (Directeur)\\
        Guillaume Blanchet\\
        Pierre Legagneux\\
        Laura Pollock\\}

        \vspace{8em}
        {Benjamin Mercier}

        18 décembre 2020

    \end{center}
\end{titlepage}


\section{Mise en contexte}
Les changements climatiques et les pressions anthropiques affectent la biodiviersité sous toutes ses formes \cite{Diaz-01SumPol}, et ce incluant les interactions entre espèces \cite{Estes2011TroDow, Purves2013TimMod, Woodward2010CliCha}. La recherche sur la prédiction de la topologie des réseaux d'interaction a largement été étudiée en comparaison avec la prédiction des forces d'interaction. Il va toutefois sans dire que la prédiction des forces d'interaction est aussi importante, sachant que cela permettrait d'améliorer notre compréhension de la dynamique des communautés \cite{Paine1992FooAna}, de la stabilité des réseaux \cite{Neutel2002StaRea, deRuiter1995EnePat} et du fonctionnement et des services écosystémiques \cite{Duffy2002Bioeco, Montoya2003FooWeb}. La force d'une interaction peut être décrite de plusieurs manières \cite{Berlow2004IntStr}, mais est généralement représentée par une fréquence d'interaction ou un flux de biomasse \cite{Heleno2014EcoNet}. L'énergie étant l'unité commune reliant tous les niveaux biologiques, représenter les forces d'interaction en tant que flux d'énergie au sein des réseaux d'interaction pourrait possiblement mener à une intégration du cadre théorique des réseaux trophiques à celui du Fonctionnement Biodiversité-Écosystème \cite{Barnes2018EneFlu}.\\

La prédiction des forces d'interaction a tout de même déjà été explorée et réalisée, et ce selon différents cadres théoriques. L'importance de la masse corporelle des organismes dans la prédiction de la force d'interaction a notamment été explicitée dans un cadre Lotka-Volterra \cite{Yodzis1992BodSiz, Pawar2015RolBod,Emmerson2004PrePre}.\citet{Brown2004MetThe} a développé la Théorie Métaboloique en Écologie qui lie la masse corporelle au taux métabolique et qui a été utilisée dans la prédiciton des forces d'interaction \cite{Berlow2009SimPre}. \cite{Brose2010BodCon} a prédit les forces d'interaction en liant la masse corporelle au cadre théorique de la Stratégie optimale de recherche de nourriture. Il est également important de mentionner que la réalisation d'une interaction dépend de la distribution local des traits des organismes et de leur abondance \cite{Poisot2015SpeWhy}.\\

Ces différentes approches ont révélé d'importants paramètres qui déterminent les forces d'interaction, mais qui pour la plupart ne représentaient pas cette force d'interaction en flux d'énergie ou étaient des modèles théoriques non validés sur des réseaux d'interaction empiriquement échantillonnés. Les modèles validés empiriquement ne représentent pas la force d'interaction comme un flux d'énergie. C'est pourquoi le développement d'un modèle représentant la force d'interaction en flux d'énergie et sa validation sur un modèle empirique est nécessaire.


\section{Objectifs}
Les principaux objectifs du projet sont:
\begin{enumerate}
    \item Développer un modèle méchanistique qui explique de manière fidèle la distribution des flux d'énergie dans un réseau d'interactions trophiques
    \item Valider le modèle sur des réseaux d'interactions trophiques empiriquement échantillonnés 
    \item Si les données le permettent, explorer comment la distribution des flux d'énergie varie selon différents écosystèmes/spatialement
\end{enumerate}

\section{Méthodologie}
Initialement, j'avais imaginer utiliser une méthode statistique de style "Likelyhood" en général, pour voir quels modèles avec quels paramètres représentent le mieux un modèle empirique, en ajoutant ou enlevant des variables et explorer d'autres variables qui pourraient expliquer la variance résiduelle. Toutefois,
les modèles que nous allons explorer ne sont pas encore définis, puisque je dois d'abord synthétiser la littérature empirique abordant les forces d'interaction prédateur-proie avant de proposer un modèle général. En parallèle, nous allons comparer des modèles méchanistique à un modèle phénoménologique (i.e. algorithme RandomForest), pour avoir une meilleure idée de ce qui est possible dans la prédiction des forces d'interaction à l'aide de données déjà disponibles telles que les traits des espèces, les abondances, la taxonomie, etc. \citet{Brose2019PreTra} ont suggéré l'utilisation de méthode semblable, telle une "boite noire", pour avoir une meilleure idée des variables ayant un rôle important dans la prédiction des forces d'interaction.\\

Pour ce faire, nous devons trouver des jeux de données de réseaux trophiques échantillonnés quantiativement avec de l'information sur les flux d'énergie, l'abondance ou la biomasse des espèces, et d'autres traits pertinents comme la masse corporelle, la taille corporelle, ;es taux métaboliques, les types de métabolismes, les vitesses de déplacement, les capacités de détection etc. Plusieurs traits importants sont potentiellement disponibles dans d'autres bases de données. En raison de la nature des données disponibles, nous allons potentiellement devoir utiliser des traits représentant la moyenne adulte d'une espèce et non des traits mesurés à l'individu, par exemple la masse corporelle adulte moyenne.\\

Ensuite, je devrai formaliser la théorie par rapport à ce qui défini les interactions entre espèces afin d'élaborer une hypothèse sur ce qui influence les forces d'interaction. Différents idées qui pourraient être explorées sont par exemple: Est-ce qu'un couplage des traits plus fort entre deux espèces mène à une force d'interaction plus grande? Est-ce que la force d'interaction entre deux espèces dépend du temps de co-occurence annuel entre ceux-ci (saisonalité)?\\


%Le style unsrtnat est bien adapté à Natbib et met la bibliographie en ordre d'apparition dans le texte et non en ordre alphabétique
\pagebreak
\bibliographystyle{unsrtnat}
\bibliography{ref}

\end{document}












\documentclass[english,12pt]{article}

%Encoding
%------------------------------------
\usepackage[utf8]{inputenc}
\usepackage[T1]{fontenc}
%------------------------------------

%Bibliography
%------------------------------------
\usepackage{babel}
\usepackage[sorting=anyt, style=alphabetic, citestyle=authoryear, natbib]{biblatex}
\addbibresource{ref.bib}

%Misc
%------------------------------------
\usepackage[onehalfspacing]{setspace} % Set the spacing between lines
\usepackage[a4paper, total={6in, 8in}]{geometry} % Size of the page
\usepackage{graphicx} % To include images and graphs
\usepackage{amsmath} % To include math equations
\usepackage[normalem]{ulem} % To underline a word
\usepackage[left]{lineno} % To add line numbers
\linenumbers % To add line numbers
%------------------------------------

\begin{document}
\begin{titlepage}
    \begin{center}

        \vspace*{-6em}
        
        \Large{\textbf{Proposé de recherche}}\\
        {\textbf{BIO700}}
        \vspace{3em}

        \vspace{3em}
        \includegraphics[scale=0.65]{udes_logo.jpg}\\
        \vspace{5em}

        \normalsize{\textbf{Présenté à:}\\
        Dominique Gravel (Directeur)\\
        Guillaume Blanchet\\
        Pierre Legagneux\\
        Laura Pollock\\}

        \vspace{15em}
        {Benjamin Mercier}

        18 mars 2021

    \end{center}
\end{titlepage}

\tableofcontents
\newpage

\begin{center}
\Large{\textbf{Empirically validated mechanistic model of energy flux inference in trophic networks}}
\end{center}

\section{Introduction}
\subsection{In context}
Biodiversity, including species interactions, are at risk of being altered in the face of climate change and anthropic stressors \citep{Estes2011TroDow,Purves2013TimMod,Woodward2010CliCha}. Notably species phenologies and local abundances might be imapcted, which could result in a temporal and spatial mismatch leading to the loss of interactions \citep{Montoya2010CliCha,Parmesan2003GloCoh,Schweiger2008CliCha,Renner2018CliCha,Miller-Rushing2010EffPhe,Visser2005ShiPhe}. Another particular effects of the aforementioned impact is trophic downgrading, which is defined by the loss of apex organisms in the environement \citep{Estes2011TroDow}. Trophic downgrading might further impact the structure and dynamics of interaction networks through trophic cascades, and disrupt how energy flows within them \citep{Estes2011TroDow,Duffy2002BioEco}. Ecosystem functioning are tightly connected to species identities, their interactions and how strongly they interact \citep{Duffy2002BioEco}. Thus the impacts on organisms forming these interactions may also possibly not be limited to the interactions themselves, but might also scale to whole ecosystem and their functioning \citep{Woodward2010CliCha,Traill2010RevMec}, which might further alter ecosystem services on which we rely as a society \citep{Dobson2006HabLos,Montoya2010CliCha}.

\subsection{Rationale and importance}
Knowing the possible consequences of the alteration in species interactions within ecological networks, it becomes obvious that understanding what drives these interactions and how they take place is crucial. The problem is that sampling of species interactions and interaction networks is a highly resource-consuming task from natural communities given simply their huge amount \citep{Jordano2016ChaEco,Hegland2010HowMon,Novak2008EstNon} which can be explained by the proportionnal increase of interactions with the square of species richness \cite{Gravel2016MeaFun}. It is important to keep in mind that absence of interactions can be due to a low probability link, low local abundances or low per-individual interaction strength \citep{Wells2013SpeInt}. Additionnally, the empirical estimation of the strength of these interactions is all the more a challenge \citep{Wootton1997EstTes,Sala2002ComDisc} notably because of possible feedback loops of indirect and density-dependent effects when measurements are taken over a too large time interval \citep{Wootton2005MeaInt}. Interactions can either be infered empirically from the field and laboratory or theoretically from models \citep{Morales-Castilla2015InfBio}. With all the limitations and difficulties of empirical sampling in mind, it becomes clear that the development of predictive model is essential to track how interactions strength will change over time in a context of climate change. It remains that there is an additional problem regarding the sampling of interaction networks. Researchers have different goals with different sampling methods \citep{Heleno2014EcoNet} which results in an heterogeneity in the disponibility and quality of data. I will further discuss this problem at the end of the proposal in the section "Call for a unified sampling of quantitative interaction networks". Nerverthless, quantitative empirical sampling of networks of interactions are still essential because empirical data are usefull for model parameterization and validation \citep{Novak2008EstNon}.\

To continue with the development of predictive model, it has in fact already been done quite extensively. To be more precise, a lot of these predictive models were predicting network topology and structure, and within many different frameworks. Without going into too much details, \cite{Pascual2005EcoNet} nicely summarise a lot of them: the Random Model \citep{cohen1977FooWeb}, the Cascade Model \citep{Cohen1985StoThe}, the Niche Model \citep{Williams2000SimRul}, the Nested-Hierarchy Model \citep{Cattin2004PhyCon}, which are all models that predict which species will interact with which species based on different sets of rules. In complement, network topology can also be infered by different kind of proxies that relates to interactions, namely their traits, their phylogeny and data on the distribution of species \citep{Morales-Castilla2015InfBio}, which is more generally called trait-matching. Different methods of machine learning have also been used to reconstruct network topology, such as a kNN algorithm \citep{Desjardins-Proulx2017EcoInt}, RandomForest and neural networks \citep{Pichler2020MacLea}. All these methods aimed at predicting network topology which is qualitative: are any set of two species interacting or not. The point here is that the developpement of predictive models for network topology have been extensively realized in contrast to the prediction of interaction strength. Even if predictive models of interaction strength have already been explored, to our knowledge none were validated with reference data. Different approaches to predict interaction strength will be further develop in the section "How to predict interaction strength". The following section will first develop general theory on interaction strength and quantitative networks.


\section{Interaction strength and quantitative networks}
\subsection{What is interaction strength}
In contrast to the topology of a network which refers to the presence or absence of interactions, interaction strength is the quantification of an interaction. Just like network topology, quantitative networks can be stored in adjacency matrices in which species are stored as columns and rows \citep{Delmas2019AnaEco,Wells2013SpeInt,Pascual2005EcoNet} where columns usually represent the consumers and rows the resources. The only distinction is that the matrices are filled with either proportions (frequencies) of interactions, which are more frequent in pollination networks and bipartite networks, or actual numbers, ranging from 0 to whatever the number depending on the definition of interaction strength employed, representing the strength of interaction. Quantification of interaction is important since they don't have the same prevalence in the environment and thus the same importance \citep{Paine1980FooWeb,Benke1997TroBas}. The added layer of information that interaction strength brings to the fondamental network topology is non trivial as it could improve our understanding of community dynamics \citep{Paine1992FooAna,Laska1998TheConb}, of ecosystem functioning \citep{Montoya2003FooWeb}, of network stability \citep{Neutel2002StaRea,deRuiter1995EnePat} and the development of multispecies models \citep{Wootton2005MeaInt}.\

Interaction strength can be calculated and reported in many different ways, for example: interaction frequency, relative prey preference, change in growth rates relative to anoter species abundance, maximum consumption rate, and many more \citep{Berlow2004IntStr,Laska1998TheConb,Wootton2005MeaInt}. It can nevertheless be grouped into two main categories: 1) the strength of an individual link between any two sets of species, and 2) the effect of the changes in a species population on another species population or on the whole network \citep{Berlow2004IntStr}. Here we focus on the first definition, which is the actual strength of interaction occuring between any two set of species. Even if there is variability in how interaction strengths is reported, one recurring observation throughout many studies is that networks are usually composed of many weak links and a few strong links \citep{Berlow2004IntStr,McCann1998WeaTro}. The distribution of interaction strength within networks has multiple consequences as it is greatly related to their stability \citep{Brose2008ForThe}, where for example the greater presence of weak interactions could notably decrease the probability of extinction and invasion \citep{Sala2002ComDisa,McCann1998WeaTro}.\

Since energy can be seen as the common currency connecting every level of biology from single organisms to whole ecosystem \citep{Brown2004MetThe,Barnes2018EneFlu}, the prediction of interaction strength as energy fluxes could potentially help bridging gaps of different spheres in ecology. Notably it could help reconcile community and network ecology to the Biodiversity-Ecosystem Functionning (BEF) framework \citep{Barnes2018EneFlu}. Furthermore, the development of energy fluxes models within network could potentially supplement and improve General Ecosystem Models (GEMs), as presented by \citep{Purves2013TimMod,Harfoot2014EmeGlo}, which aim at mechanistically modelling whole ecosystems. This project especially aims at being the first step of the developpement of a General Ecosystem Model by trying to develop a model that predict energy fluxes between organisms which could later on be implemetend into a bigger GEM. For clarity purpose, energy fluxes here are meant to be seen as a flow of carbon or biomass from one species to another, per area per time.

\subsection{What affect interaction strength}
Interaction between any set of species can either be direct or indirect \cite{Morales-Castilla2015InfBio,Schmitz2001EffTop}. In the case of this project, we focus on trophic interactions which are considered direct interactions, and depends on a multitude of parameters to actually take place. Species interactions and their strength are susceptible to changes in the biotic environement \citep{Tylianakis2008GloCha}. Species abundances are also known to have an effect on interactions where more abundant species have greater chance to cross path and interact \citep{Bartomeus2016ComFra}, and also on the strength and symetry of interactions, where for example rare specialists would have a weak effect on an abundant species \citep{Canadard2014EmpEva,Vazquez2007SpeAbu}. Furthermore, the composition of the community and the distribution of traits locally can also intervene in the realization of interactions \citep{Poisot2015SpeWhy}. Some particular traits of organisms are especially important in the realization of interactions like morphological, physiological, phenological and behavioural \citep{Morales-Castilla2015InfBio}, or more specifically to predict interactions. What is great about the prediction of energy fluxes is that it has the potentiel to be well predicted with the use of species basic traits such as their body sizes, abundances and metabolic rates \citep{Berlow2004IntStr}. Different types of traits are more relevant to predict different types of interactions, whereas for example morphological and physiological traits might be more important for predation interaction \cite{Bartomeus2016ComFra}. Other species traits or parameters can also be used to predict energy fluxes, as we will see in the two following sections.\

\section{How to predict interaction strength and energy fluxes}
Quite a few different models have already been used to predict interaction strength, but not all were predicting interaction strength as a flux of energy. Here I will go over a few of them, which are the ones in my opinion that are the most relevant, and point out some of their parameters that could potentially be usefull for energy fluxes predictions.

The Lotka-Volterra framework was used quite a lot to predict interaction strength, notably by \cite{Yodzis1992BodSiz}. It usually models the change of the abundance or biomass of a species over time with differential equations, based on different parameters like intrinsic biomass growth, metabolic rates, biomass conversion efficiency which in these cases are not related to species traits but are arbitrary constants \citep{Williams2006HomYod}, which makes these models more phenomenological. As \cite{Harfoot2014EmeGlo,Purves2013TimMod} argue, there is a dire need of mechanistic models to understand comprehensively the state of ecosystems and how each part of it (biological, physiological, ecological etc.) behave. In contrast to phenomenological models, mechanistic models explicitly express the state of a system as its different components and how they operate and interact with one another to describe a phenomenon \citep{Connolly2017ProMec}, which is in return also more suited for prediction \citep{Ings2009RevEco}. In our specific case, the "phenomenon" is the transfer of energy through fluxes which is expressed by different components that are different species and environment traits. 

In the following section I present some models that predicted energy fluxes from a prey to a predator. 
As \cite{Portalier2019MecPre} pointed out, statistical models which predict interactions are well suited for prediction on network that are simi;lar to the ones the model was built from, thus making a mechanistic approaches to prediction more desirable because of their more general applicability. Mechanistic models which are trait-based already existed but were either disconnected from real networks or were incorporating traits that were too species specific. \cite{Portalier2019MecPre} thus proceeded to make what he called a Newtonian mechanical approach where he predicts the energy gains (J/kg) of a predation action, by expliciting the actions of searching, capturing and handling, based on lower-level organisms traits and environmental traits. The principal traits that were used are body mass, wich are related with metabolic law and physical traits related to the environment like dimensionality. Even though this model aims at predicting which interactions are feasable in term of energetic expenditures to predict the basic network topology, it explicits that these traits could be useful/important for prediction of interaciton strength.

\cite{Brown2004MetThe} developped the Metabolic Theory of Ecology where they established that the metabolic rate of an organism scales with their body mass to the 3/4 power-law. This framework was later on developped with further organisms traits. As an example, \cite{Pawar2012DimCon} was able to come up with relation between consumption rates that were scaled with body mass, where the exponent varied based on the dimensionality of the interactions: if the interactions are considered 2D the exponent is 0.78 and if it is considered 3D the exponent is 1.16, which reveals that interaction strength will probably vary with dimensionality since consumption rate does. 

\cite{Brose2008ForThe} compared a metabolic model to a foraging model where he used the Metabolic Theory of Ecology developped by \cite{Brown2004MetThe}. His models utilize species body masses, 

\cite{Barnes2018EneFlu} used a food-web energetics approach, where they predicted the flux of energy based on species assimilation efficiency, their metabolic demand and the loss of energy from predation of higher trophic level. To do so, the main species traits needed were body masses that were used to calculate the metabolic demands. There were three possible ways to get the assimilation effiency in the study: 1) they were either measured, 2) obtained from the litterature by consumer type or temperature or 3) could be scaled with resource stoichiometry.

\cite{Gauzens2019FluRP} also developped an approach to estimate energy fluxes, \emph{fluxweb}, for whole networks using a top-down approach where the fluxes are estimated from the higher trophic-level down to the basal one. To do so, their model is based on 4 main parameters which are : 1) the interactions themselves (who interact with who), 2) the physiological losses which can be represented as metabolic rates (as presented by \cite{Brown2004MetThe}), 3) the feeding efficiencies, which will vary depending on the resource type and 4) the species population total biomass. The physiological loss parameters represented as metabolic rates can vary on the organism types as reported in \cite{Brown2004MetThe}:\
\begin{equation*}
  X_{i} = x_{0}*M_{i}^{b}
\end{equation*}
where the different parameter values for $x_{0}$ and \emph{b} are presented in table 1.
\

\begin{table}[h!]
\centering
\begin{tabular}{ |c|c|c| }
 \hline
 Metabolic type & $x_{0}$ & \emph{b} \\ 
 \hline
 Ectotherm vertebrates & 18.18 & -0.29 \\ 
 Endotherm vertebrates & 19.5 & -0.29 \\
 Invertebrates & 17.17 & -0.29 \\
 \hline
\end{tabular}
\caption{Metabolic rate parameter values depending on organisms metabolic type. Reproduced from \cite{Gauzens2019FluRP} based on \cite{Brown2004MetThe}.}
\label{tab_meta}
\end{table}\

Overall, the main traits needed in the \emph{fluxweb} model are species body mass, species metabolic types and species total population biomass.

The foraging theory of ecology generally describes the consumption rates of consumer towards their resources, giving a rate of consumption wich initially relied mainly on consumer and resource abundances \citep{brose2008}. 

\subsection{Useful traits/data for prediction}
The distinction of metabolic type among organism is important for body size to be a good predictor of metabolic and maximum assimilation rates \citep{Williams2006HomYod}. Organisms can then be modelled as biomass stock that shrinks due to predation and metabolic demands, and grows from predation (for predator) and net growth from producers \citep{Williams2006HomYod}. 

- validation of theoretical model on empirical data


Since network perturbations can be felt in many trophic level within a network, the reconstitution of energy flows should be made to encompass the whole network \citep{Delmas2017SimBio}, and not simply between pairwise interactions.


\section{Goals and hypothesis}
Bulding on what was previously stated in the previous section, which is that predictive models of interaction strength were usually not predicting energy fluxes, were not mechanistic or were not validated over reference data, the main objectives of the project are:
\begin{enumerate}
    \item Develop a mechanistic model that accurately explains the distribution of energy fluxes within trophic networks
    \item Validate it on empirically sampled quantitative networks 
    \item If the data allows it, explore how the distribution of energy fluxes varies amongst different ecosystems/spatially
\end{enumerate}

No hypotheses per se are yet defined.


\section{Methodology}
In parallel, we will compare mechanistic models to a phenomenological one (e.g. random forest algorithm) to get an idea of how far we could go at predicting interactions with the available information (traits, abundances, taxonomy). \citet{Brose2019PreTra} suggested to do so while including more variables, in kind of a "black-box" approach to see what ends up being important in predicting interaction strength. To do so we need to find quantitative food web datasets with information on energy fluxes, abundance or biomass, and other traits such as body mass, body size, metabolic rates, metablosim types, movement speed, detection capacity etc. Most of the other useful traits for prediction can be found in other databases. Because of the nature of data available, we will probably have to use traits average to the average species-level for example average adult body mass. 

The model I wish to develop is a simplification of actual trophic networks since trophic interactions can be affected by other non-trophic processes, for example: interference competition, facilitation and environmental stresses which in the end have an impact on species abundances \citep{Berlow2004IntStr}. So the model is probably a best case scenario in which only the action of predation between two species is happening, which will result in the maximal potential biomass flux between them.

If 

\subsection{The data}
As previously said, data on quantitatively sampled trophic networks are quite scarce. The initial idea was to use empirically sampled trophic network, but since there aren't a lot of them publicly available it forces us to go in a different way. Ecopath models are probably our best bet right now since there is quite a lot of available models, they span over different kind of ecosystem (i.e. marine, aquatic and terrestrial) and they are all built within the same framework. We will first start with the 116 food webs used in the study of \citet{Jacquet2016NoCom}, and if needed will incorporate more along the way. One particular characteristic of Ecopath networks is that they encompass a lot of trophic group to the detriment of individual species. This will need to be addressed as the model we aim to develop will have parameters that are defined at the species level such as average adult body mass for example. One reason to work at the species-level is that taxa lumped into a trophic group don't interact the same way with all the species constituting the trophic group \citep{Ings2009RevEco}. Thus all networks will have to be skimmed of the interactions that are where at least one of the two members of the interaction is a trophic group or lumped species. Some trophic group will probably be able to be kept since they are rarely represented as an individual species and generic parameters value could still make sense for example phytoplankton. We will also have to be careful for our model to not be circual with how Ecopath models are constructed, because if any circularity is made between the two models, validation will make little to no sense. Seasonlity will have to be adressed in some kind of way in the model or at least be mentionned. Seasons may have an impact on species abundances \citep{Ings2009RevEco}, community compositions \citep{Mellard2019SeaPat}, and how they forage and thus might play a role in the variance of energy fluxes \citep{McMeans2019ConTro}. Ideally we would have to work with networks that experience little seasonality or at least mention that seasonality was not taken into account. Furthermore, I would still like to try and validate the model on empirically sampled trophic networks to see if there is any major differences with the Ecopath models. 


\subsection{Mechanistic models} 
In contrast with phenomenological models, mechanistic models provide more accurate description of what is really happening ecosystem-wise \cite{Delmas2017SimBio}. 
Non-linear functional response that saturate the consumption of predators on preys can either be prey-dependent, predator-dependent or ratio-dependent \cite{Williams2006HomYod}.
**Need a assimilation efficiency**
e = 0.5 \citep{Pawar2015RolBod}.
\begin{equation*}
  \phi_{ij} = \varepsilon_{ij}*B_{i}*B_{j}
\end{equation*}


\section{Energy fluxes prediction MISC}
Growth rates, body sizes and more herbivory (in aquatic system) govern how energy flows within food webs \citep{Rip2011CroDif}. Primary producers from aquatic ecosystem usually have a higher growth rates than their terrestrial counterparts, and it should result in a inverted biomass pyramid which deacreses the stability \citep{Rip2011CroDif}. The maxium growth rates scales with body mass as a -0.25 exponent \citep{Rip2011CroDif} citing \citep{Brown2004MetThe}.\

The use of biomass, instead of abundance, reflects species consumption and also more broadly ecosystem parameters such as biodiversity and ecosystem functionning \citep{Emmerson2004PrePre}, which might come handy in some specific frameworks. Furthermore, abundance is more linked to some population processes such as birth and death which can greatly vary in time wich can affect interaction strength, making a biomass approach more favorable \citep{Wootton2005MeaInta}.

Interactions between species can be defined by the foraging traits of the consumer and the vulnerability traits of the resources \citep{Laigle2018SpeTra}.

In the model that \citet{Yodzis1992BodSiz} developped, interactions depend on 5 principal biological factors which are: the metabolic type, the type of functionnal response, the resource abundance, the ecological limitations on resource acquisition and the relative rates of consumer/resource consumption.

According to \citet{Pawar2015RolBod}, body size is a key trait because it can determine the strength of interspecific trophic interactions and also life history rates which determine population energetics (metabolism). Body size also usually increase with trophic level.
 
Allometric model based on the metabolic scalling theory suggest that consumption rates of predators follow a power-law scaling with body mass \citep{Brose2008ForThe}.

\citet{Berlow2009SimPre} obtain prediction of interaction strength (removal style) with simple functions of species biomass and body size.

\subsection{Call for unified empirical sampling of quantitative interaction networks}
This section isn't really a part of the research project itself but rather an idea I had while trying to find and format the data needed for the project. I quickly realized that open data of empirically sampled quantitative trophic networks were a rare commodity. Either I didn't know where to look, which I believe isn't the case because I am \uline{relatively} familiar with network databases and searching within articles for openly available data, either such data exists but are not yet shared openly to the research community, or finally either there is just not a lot of trophic networks that were empirically sampled. I want to point out that quite a lot of quantitative trophic networks are available openly, but the majority are not empirically sampled but rather prediction of different kind of models, such as EcoPath modelisation. While such data serve a purpose, they aren't suited for model validation like empirical data.\

Over two decades ago, \cite{Cohen1993ImpFooa} pointed out the lack of standardization on how trophic networks were sampled and reported. Still to this day, the ways in which trophic networks and strengths in those networks are reported is quite heterogenous. As summarized by \cite{Berlow2004IntStr} and \cite{Laska1998TheConb}, there exists a multitude of ways to represent interaction strength usually depending on the researchers end goals. It is quite common to encounter data that have been lumped together such as trophic or functional groups which is a tradeoff on network realism \citep{Heleno2014EcoNet}. \cite{Schmitz2001EffTop} found that the grouping of similar functional species into groups might actually oversimplify the dynamics of community, thus probably missing important biological mechanims. As \cite{Heleno2014EcoNet} stated, our simplification of nature in ecological networks need to be based on a strong scientific foundation so further analysis of networks constructed by different researchers are conceivable and accurate. \cite{Wells2013SpeInt} suggests that a lot of networks are established by aggregating data over time and space and are used to do network metric analysis which could be prone to biases from sample sizes related to sampling methods making such networks useless for comparative studies. Thus, sampling designs of ecological networks should be made very clear to ease the comparisons between studies.

In this context, I would like to suggest an homogenization of how interaction strength is reported and calculated to try and make different studies more comparable.
In retrospective, I feel like there is still a need for a kind of unification of how we see and report interaction strength. I would then like to write a little piece on the matter. Any thoughts on this idea are more than welcome, as if it is a good idea, anyone who might be interested in contributing etc.

\pagebreak
\printbibliography

\end{document}
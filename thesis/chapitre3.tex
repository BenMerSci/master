\chapter{\textbf{RÉSULTATS / ARTICLE SCIENTIFIQUE}} 

\section{Titre}
\lipsum[10]
\vfill{}
\pagebreak

\begin{center}

\textbf{Titre de l'article} \\
\textit{Journal} (Année), Volume (Issue): pages. \\
Auteurs 
\end{center}

\section{Titre}
Exemple de références qui n'apparaît que dans ce chapitre : \cite{Stearns1992}. 

\lipsum[2]

\vfill{}
\textbf{Keywords :} 
\pagebreak

\section{Titre}

Faire référence aux tableau \ref{ch3table1} au moins une fois et faire référence au matériel supplémentaire en annexe au moins une fois : Figure \ref{ch3FigS1} de l'Annexe B. 


% latex table generated in R 3.5.1 by xtable 1.8-3 package
% Mon Apr  8 23:23:20 2019
\begin{table}[ht]
\centering
\caption[Titre de tableau]{\label{ch3table1} Titre de tableau} % le premier titre va dans la liste de tableaux sans espace 'Tab' supplémentaire 
\resizebox{0.95\linewidth}{!}{%
% the % sign add space before and after table
\begin{tabular}{lcccccccccccccc}
  \toprule
 & & \multicolumn{6}{@{}c}{{}Univariate linear mixed effects models}& \multicolumn{6}{@{}c}{{}} \\ 
  \midrule
Milk components & & $V_{Individual}$ & 95\% CI & $V_{Residual}$ & 95\% CI & $R2_{Conditional}$ & R\\% 
  Proteins  &  & 0.02  & 0.00 - 0.08 & 0.67  & 0.54 - 0.80 & 0.08  & 0.03\\%
    \bottomrule
\end{tabular} %
}
\medskip
\captionsetup{labelformat=empty}
\caption[]{\qquad\qquad\quad\ \normalfont{Légende de tableau}} % \qquad ou \quad pour ajouter des espaces
\end{table}




\lipsum[2]



%-------------------BIBLIOGRAPHIE----------------------------
% assurez vous d’avoir un dossier avec vos listes de références par chapitre. C’est ici que chaque liste est intégrée au chapitre. 
% il faut la commande \bibliography ET \bibliographystyle pour que la liste par chapitre (avec différents styles) fonctionne. 

\singlespacing
{\renewcommand{\bibname}{References}
\renewcommand{\bibsection}{\section{\bibname}}
\bibliography{bib/chapitre3}}
\bibliographystyle{styles/myBEAS} 

